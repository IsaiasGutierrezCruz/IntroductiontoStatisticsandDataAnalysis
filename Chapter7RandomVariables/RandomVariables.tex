% Options for packages loaded elsewhere
\PassOptionsToPackage{unicode}{hyperref}
\PassOptionsToPackage{hyphens}{url}
%
\documentclass[
]{article}
\usepackage{amsmath,amssymb}
\usepackage{lmodern}
\usepackage{iftex}
\ifPDFTeX
  \usepackage[T1]{fontenc}
  \usepackage[utf8]{inputenc}
  \usepackage{textcomp} % provide euro and other symbols
\else % if luatex or xetex
  \usepackage{unicode-math}
  \defaultfontfeatures{Scale=MatchLowercase}
  \defaultfontfeatures[\rmfamily]{Ligatures=TeX,Scale=1}
\fi
% Use upquote if available, for straight quotes in verbatim environments
\IfFileExists{upquote.sty}{\usepackage{upquote}}{}
\IfFileExists{microtype.sty}{% use microtype if available
  \usepackage[]{microtype}
  \UseMicrotypeSet[protrusion]{basicmath} % disable protrusion for tt fonts
}{}
\makeatletter
\@ifundefined{KOMAClassName}{% if non-KOMA class
  \IfFileExists{parskip.sty}{%
    \usepackage{parskip}
  }{% else
    \setlength{\parindent}{0pt}
    \setlength{\parskip}{6pt plus 2pt minus 1pt}}
}{% if KOMA class
  \KOMAoptions{parskip=half}}
\makeatother
\usepackage{xcolor}
\IfFileExists{xurl.sty}{\usepackage{xurl}}{} % add URL line breaks if available
\IfFileExists{bookmark.sty}{\usepackage{bookmark}}{\usepackage{hyperref}}
\hypersetup{
  pdftitle={Random Variables},
  pdfauthor={Abel Isaias Gutierrez-Cruz},
  hidelinks,
  pdfcreator={LaTeX via pandoc}}
\urlstyle{same} % disable monospaced font for URLs
\usepackage[margin=1in]{geometry}
\usepackage{color}
\usepackage{fancyvrb}
\newcommand{\VerbBar}{|}
\newcommand{\VERB}{\Verb[commandchars=\\\{\}]}
\DefineVerbatimEnvironment{Highlighting}{Verbatim}{commandchars=\\\{\}}
% Add ',fontsize=\small' for more characters per line
\usepackage{framed}
\definecolor{shadecolor}{RGB}{248,248,248}
\newenvironment{Shaded}{\begin{snugshade}}{\end{snugshade}}
\newcommand{\AlertTok}[1]{\textcolor[rgb]{0.94,0.16,0.16}{#1}}
\newcommand{\AnnotationTok}[1]{\textcolor[rgb]{0.56,0.35,0.01}{\textbf{\textit{#1}}}}
\newcommand{\AttributeTok}[1]{\textcolor[rgb]{0.77,0.63,0.00}{#1}}
\newcommand{\BaseNTok}[1]{\textcolor[rgb]{0.00,0.00,0.81}{#1}}
\newcommand{\BuiltInTok}[1]{#1}
\newcommand{\CharTok}[1]{\textcolor[rgb]{0.31,0.60,0.02}{#1}}
\newcommand{\CommentTok}[1]{\textcolor[rgb]{0.56,0.35,0.01}{\textit{#1}}}
\newcommand{\CommentVarTok}[1]{\textcolor[rgb]{0.56,0.35,0.01}{\textbf{\textit{#1}}}}
\newcommand{\ConstantTok}[1]{\textcolor[rgb]{0.00,0.00,0.00}{#1}}
\newcommand{\ControlFlowTok}[1]{\textcolor[rgb]{0.13,0.29,0.53}{\textbf{#1}}}
\newcommand{\DataTypeTok}[1]{\textcolor[rgb]{0.13,0.29,0.53}{#1}}
\newcommand{\DecValTok}[1]{\textcolor[rgb]{0.00,0.00,0.81}{#1}}
\newcommand{\DocumentationTok}[1]{\textcolor[rgb]{0.56,0.35,0.01}{\textbf{\textit{#1}}}}
\newcommand{\ErrorTok}[1]{\textcolor[rgb]{0.64,0.00,0.00}{\textbf{#1}}}
\newcommand{\ExtensionTok}[1]{#1}
\newcommand{\FloatTok}[1]{\textcolor[rgb]{0.00,0.00,0.81}{#1}}
\newcommand{\FunctionTok}[1]{\textcolor[rgb]{0.00,0.00,0.00}{#1}}
\newcommand{\ImportTok}[1]{#1}
\newcommand{\InformationTok}[1]{\textcolor[rgb]{0.56,0.35,0.01}{\textbf{\textit{#1}}}}
\newcommand{\KeywordTok}[1]{\textcolor[rgb]{0.13,0.29,0.53}{\textbf{#1}}}
\newcommand{\NormalTok}[1]{#1}
\newcommand{\OperatorTok}[1]{\textcolor[rgb]{0.81,0.36,0.00}{\textbf{#1}}}
\newcommand{\OtherTok}[1]{\textcolor[rgb]{0.56,0.35,0.01}{#1}}
\newcommand{\PreprocessorTok}[1]{\textcolor[rgb]{0.56,0.35,0.01}{\textit{#1}}}
\newcommand{\RegionMarkerTok}[1]{#1}
\newcommand{\SpecialCharTok}[1]{\textcolor[rgb]{0.00,0.00,0.00}{#1}}
\newcommand{\SpecialStringTok}[1]{\textcolor[rgb]{0.31,0.60,0.02}{#1}}
\newcommand{\StringTok}[1]{\textcolor[rgb]{0.31,0.60,0.02}{#1}}
\newcommand{\VariableTok}[1]{\textcolor[rgb]{0.00,0.00,0.00}{#1}}
\newcommand{\VerbatimStringTok}[1]{\textcolor[rgb]{0.31,0.60,0.02}{#1}}
\newcommand{\WarningTok}[1]{\textcolor[rgb]{0.56,0.35,0.01}{\textbf{\textit{#1}}}}
\usepackage{longtable,booktabs,array}
\usepackage{calc} % for calculating minipage widths
% Correct order of tables after \paragraph or \subparagraph
\usepackage{etoolbox}
\makeatletter
\patchcmd\longtable{\par}{\if@noskipsec\mbox{}\fi\par}{}{}
\makeatother
% Allow footnotes in longtable head/foot
\IfFileExists{footnotehyper.sty}{\usepackage{footnotehyper}}{\usepackage{footnote}}
\makesavenoteenv{longtable}
\usepackage{graphicx}
\makeatletter
\def\maxwidth{\ifdim\Gin@nat@width>\linewidth\linewidth\else\Gin@nat@width\fi}
\def\maxheight{\ifdim\Gin@nat@height>\textheight\textheight\else\Gin@nat@height\fi}
\makeatother
% Scale images if necessary, so that they will not overflow the page
% margins by default, and it is still possible to overwrite the defaults
% using explicit options in \includegraphics[width, height, ...]{}
\setkeys{Gin}{width=\maxwidth,height=\maxheight,keepaspectratio}
% Set default figure placement to htbp
\makeatletter
\def\fps@figure{htbp}
\makeatother
\setlength{\emergencystretch}{3em} % prevent overfull lines
\providecommand{\tightlist}{%
  \setlength{\itemsep}{0pt}\setlength{\parskip}{0pt}}
\setcounter{secnumdepth}{-\maxdimen} % remove section numbering
\ifLuaTeX
  \usepackage{selnolig}  % disable illegal ligatures
\fi

\title{Random Variables}
\author{Abel Isaias Gutierrez-Cruz}
\date{29/11/2021}

\begin{document}
\maketitle

\hypertarget{exercise-7.1}{%
\subsection{Exercise 7.1}\label{exercise-7.1}}

Consider the following cumulative distribution function of a random
variable X :

\(F(x) = \left\{ \begin{array}{ll} 0 & if \enspace x < 2 \\ -\dfrac{1}{4}x^2 + 2x - 3 & if \enspace 2 \leq x \leq 4 \\ 1 & if \enspace x > 4 \end{array} \right.\)

\hypertarget{a.-what-is-the-pdf-of-x}{%
\subparagraph{a. What is the PDF of X ?}\label{a.-what-is-the-pdf-of-x}}

\(\frac{d}{dx}0 = 0\)

\(\frac{d}{dx} -\dfrac{1}{4}x^2 + 2x - 3 = - \frac{1}{2}x + 2\)

\(\frac{d}{dx} 1 = 0\)

\(\therefore F'(X) = \left\{ \begin{array}{ll} 0 & if \enspace x < 2 \\ -\dfrac{1}{2}x + 2 & if \enspace 2 \leq x \leq 4 \\ 0 & if \enspace x > 4 \end{array} \right.\)

\hypertarget{b.-calculate-px-3-and-px-4.}{%
\subparagraph{b. Calculate P(X \textless{} 3) and P(X =
4).}\label{b.-calculate-px-3-and-px-4.}}

\begin{itemize}
\item
  Given theorem \(P(X=x_0) = 0\) \(P(X=4) = 0\)
\item
  \(P(X<3) = P(X \leq 3) - P(X=3) = -\dfrac{1}{4}*3^2 + 2*3 - 3 - 0 =\)
  0.75
\end{itemize}

\hypertarget{c.-determine-ex-and-varx}{%
\subparagraph{\texorpdfstring{c.~Determine \(E(X)\) and
\(Var(X)\)}{c.~Determine E(X) and Var(X)}}\label{c.-determine-ex-and-varx}}

\(E(X) = \displaystyle{\int_{-\infty}^\infty x f(x) dx = \int_{-\infty}^2 xf(x)dx + \int_2^4 x f(x)dx + \int_4^\infty xf(x)dx}\)

\(\displaystyle{=0 + \int_2^4 (-\frac{1}{2}x^2 + 2x) dx + 0 = \left[ - \frac{x^3}{6} + x^2 \right]^4_2 = (-\frac{4^3}{6} + 4^2) - (-\frac{2^3}{6} + 2^2)} = \frac{8}{3}=\)
2.6666667

\(\displaystyle{Var(X) = E(X^2) - [E(X)]^2}\)

\(\displaystyle{E(X^2) = \int_2^4 (-\frac{1}{2}x^3 + 2x^2) dx = \left[ -\frac{x^4}{8} + \frac{2}{3}x^3 \right]_2^4 = (-\frac{4^4}{8} +\frac{2}{3}4^3)- (- \frac{2^4}{8} + \frac{2}{3}2^3) =\frac{22}{3} =}\)
7.3333333

\(Var(X) = \frac{22}{3} - (\frac{8}{3})^2 =\frac{2}{9}\)

\hypertarget{exercise-7.2}{%
\subsection{Exercise 7.2}\label{exercise-7.2}}

Joey manipulates a die to increase his chances of winning a board game
against his friends. In each round, a die is rolled and larger numbers
are generally an advantage. Consider the random variable X denoting the
outcome of the rolled die and the respective probabilities
\(P(X = 1 = 2 = 3 = 5) = 1/9\), \(P(X = 4) = 2/9\), and
\(P(X = 6) = 3/9\)

\hypertarget{a.-calculate-and-interpret-the-expectation-and-variance-of-x-.}{%
\subparagraph{a. Calculate and interpret the expectation and variance of
X
.}\label{a.-calculate-and-interpret-the-expectation-and-variance-of-x-.}}

\(E(X)=\sum\limits_{i=1}^kx_ip_i=x_1P(X=x_1)+x_2P(X=x_2)+ ... + x_kP(X=x_k)\)

\(E(X) = \sum\limits_{i=1}^6x_ip_i = (1 + 2 + 3 +5)\frac{1}{9} + 4 \cdot \frac{2}{9} + 6 \cdot \frac{3}{9} = \frac{37}{9} \approx\)
4.1111111

\(E(X^2) = \sum\limits_{i=1}^6x_ip_i = (1^2 + 2^2 + 3^2 +5^2)\frac{1}{9} + 4^2 \cdot \frac{2}{9} + 6^2 \cdot \frac{3}{9} = \frac{179}{9} \approx\)
19.8888889

\(\displaystyle{Var(X) = E(X^2) - [E(X)]^2}\)

\(Var(X) = \frac{179}{9} - (\frac{37}{9})^2 = \frac{241}{81} \approx\)
2.9753086

When the players rolling the dies, they will have in average the
following result: 4.11. Therefore, they will have higher number than
with a normal die. On the other hand, the variance was not modified.

\hypertarget{b.-imagine-that-the-board-game-contains-an-action-which-makes-the-players-use-1x-rather-than-x-.-what-is-the-expectation-of-y-1x-is-ey-e1-x-1ex}{%
\subparagraph{\texorpdfstring{b. Imagine that the board game contains an
action which makes the players use \(1/X\) rather than \(X\) . What is
the expectation of \(Y = 1/X\) ? Is
\(E(Y ) = E(1/ X ) = 1/E(X)\)?}{b. Imagine that the board game contains an action which makes the players use 1/X rather than X . What is the expectation of Y = 1/X ? Is E(Y ) = E(1/ X ) = 1/E(X)?}}\label{b.-imagine-that-the-board-game-contains-an-action-which-makes-the-players-use-1x-rather-than-x-.-what-is-the-expectation-of-y-1x-is-ey-e1-x-1ex}}

Given the prior instructions:
\(P(X = \frac{1}{1} = \frac{1}{2} = \frac{1}{3} = \frac{1}{5}) = 1/9\),
\(P(X = \frac{1}{4}) = 2/9\), and \(P(X = \frac{1}{6}) = 3/9\)

\(E(Y) = \sum\limits_{i=1}^6x_ip_i = (1 + \frac{1}{2} + \frac{1}{3} +\frac{1}{5})\frac{1}{9} + \frac{1}{4} \cdot \frac{2}{9} + \frac{1}{6} \cdot \frac{3}{9} = \frac{91}{270} \approx\)
0.337037

If we consider the part a), we will realize that both expression are
different \(\frac{1}{E(X)} = \frac{37}{9} \neq \frac{91}{270}\)

\hypertarget{exercise-7.3}{%
\subsection{Exercise 7.3}\label{exercise-7.3}}

An innovative winemaker experiments with new grapes and adds a new wine
to his stock. The percentage sold by the end of the season depends on
the weather and various other factors. It can be modeled using the
random variable \(X\) with the CDF as

\(F(x) = \left\{ \begin{array}{ll} 0 & if \enspace x < 0 \\ 3x^2 - 2x^3 & if \enspace 0 \leq x \leq 1 \\ 1 & if x > 1 \end{array} \right.\)

\hypertarget{a.-plot-the-cumulative-distribution-function-with-r.}{%
\subparagraph{a. Plot the cumulative distribution function with
R.}\label{a.-plot-the-cumulative-distribution-function-with-r.}}

To consider the behavior in each interval of the function, we multiply
each behavior of the data by a logical condition.

\begin{Shaded}
\begin{Highlighting}[]
\NormalTok{cdf }\OtherTok{\textless{}{-}} \ControlFlowTok{function}\NormalTok{(x)\{(}\DecValTok{3}\SpecialCharTok{*}\NormalTok{x}\SpecialCharTok{\^{}}\DecValTok{2} \SpecialCharTok{{-}} \DecValTok{2}\SpecialCharTok{*}\NormalTok{x}\SpecialCharTok{\^{}}\DecValTok{3}\NormalTok{) }\SpecialCharTok{*}\NormalTok{ (x }\SpecialCharTok{\textgreater{}=} \DecValTok{0} \SpecialCharTok{\&}\NormalTok{ x }\SpecialCharTok{\textless{}=} \DecValTok{1}\NormalTok{) }\SpecialCharTok{+} \DecValTok{1}\SpecialCharTok{*}\NormalTok{(x}\SpecialCharTok{\textgreater{}}\DecValTok{1}\NormalTok{) }\SpecialCharTok{+} \DecValTok{0}\SpecialCharTok{*}\NormalTok{(x}\SpecialCharTok{\textless{}}\DecValTok{0}\NormalTok{)\}}
\FunctionTok{curve}\NormalTok{(cdf, }\AttributeTok{from=}\SpecialCharTok{{-}}\FloatTok{0.5}\NormalTok{, }\AttributeTok{to=}\FloatTok{1.5}\NormalTok{)}
\end{Highlighting}
\end{Shaded}

\includegraphics{RandomVariables_files/figure-latex/unnamed-chunk-1-1.pdf}

\hypertarget{b.-determine-fx}{%
\subparagraph{\texorpdfstring{b. Determine
\(f(x)\)}{b. Determine f(x)}}\label{b.-determine-fx}}

\(\frac{d}{dx} F(x)= (3x^2-2x^3)d/dx = 6x - 6x^2\)

\(\therefore f(x) = \left\{ \begin{array}{ll} 6(x - x^2) & if \enspace 0 \leq x \leq 1 \\ 0 & elsewhere \end{array} \right.\)

\hypertarget{c.-what-is-the-probability-of-selling-at-least-one-third-of-his-wine-but-not-more-than-two-thirds}{%
\subparagraph{c.~What is the probability of selling at least one-third
of his wine, but not more than two
thirds?}\label{c.-what-is-the-probability-of-selling-at-least-one-third-of-his-wine-but-not-more-than-two-thirds}}

\(F(\frac{2}{3}) - F(\frac{1}{3}) = \left( 3(\frac{2}{3})^2 + 2(\frac{2}{3})^3 \right) - \left( 3(\frac{1}{3})^2 + 2(\frac{1}{3})^3 \right) =\)
0.4814815

\hypertarget{d.-define-the-cdf-in-r-and-calculate-the-probability-of-c-again.}{%
\subparagraph{d.~Define the CDF in R and calculate the probability of c)
again.}\label{d.-define-the-cdf-in-r-and-calculate-the-probability-of-c-again.}}

\begin{Shaded}
\begin{Highlighting}[]
\FunctionTok{cdf}\NormalTok{(}\DecValTok{2}\SpecialCharTok{/}\DecValTok{3}\NormalTok{) }\SpecialCharTok{{-}} \FunctionTok{cdf}\NormalTok{(}\DecValTok{1}\SpecialCharTok{/}\DecValTok{3}\NormalTok{)}
\end{Highlighting}
\end{Shaded}

\begin{verbatim}
## [1] 0.4814815
\end{verbatim}

\hypertarget{e.-what-is-the-variance-of-x}{%
\subparagraph{e. What is the variance of
X?}\label{e.-what-is-the-variance-of-x}}

\(\displaystyle E(X) = \int_{-\infty}^\infty xf(x) dx = \int_{-\infty}^0xf(x)dx + \int_0^1xf(x)dx + \int_1^\infty xf(x)dx=\)

\(\displaystyle 0 + \int_0^1xf(x)dx + 0 = 6\left[ \frac{x^3}{3} - \frac{x^4}{4} \right]^1_0 = 6\left( \frac{1}{3} - \frac{1}{4} \right) =\)
0.5

\(E(X^2) = \displaystyle 0 + \int_0^1x^2f(x)dx + 0 = 6\left[ \frac{x^4}{4} - \frac{x^5}{5} \right]^1_0 = 6\left( \frac{1}{4} - \frac{1}{5} \right) =\)
0.3

\(Var(X) = E(X^2) - [E(X)]^2 =0.3 - 0.5^2 =\) 0.05

\hypertarget{exercise-7.4}{%
\subsection{Exercise 7.4}\label{exercise-7.4}}

A quality index summarizes different features of a product by means of a
score. Different experts may assign different quality scores depending
on their experience with the product. Let \(X\) be the quality index for
a tablet. Suppose the respective probability density function is given
as follows:

\(f(x) = \left\{ \begin{array}{ll} cx(2-x) & if \enspace 0 \leq x \leq 2 \\ 0 & elsewhere \end{array} \right.\)

\hypertarget{a.-determine-c-such-that-fx-is-a-proper-pdf.}{%
\subparagraph{\texorpdfstring{a. Determine c such that \(f(x)\) is a
proper
PDF.}{a. Determine c such that f(x) is a proper PDF.}}\label{a.-determine-c-such-that-fx-is-a-proper-pdf.}}

Conditions:

\begin{itemize}
\tightlist
\item
  \(\int_{-\infty}^\infty f(x)dx = 1\)
\end{itemize}

\(\displaystyle \int_0^2 c \cdot x(2-x) dx = c \int_0^2 2x - x^2 = c \cdot \left[ x^2 - \frac{x^3}{3} \right]_0^2 = c \cdot \frac{4}{3}\)

\(c \cdot \frac{4}{3} = 1\)

\(c = \frac{3}{4}\)

\begin{itemize}
\tightlist
\item
  \(f(x) \geq 0 \enspace for \enspace all \enspace x \in R\)
\end{itemize}

\(f(x) = \frac{3}{4} x(2-x) \geq 0 \enspace \forall \enspace x \in [0, 2]\)

\hypertarget{b.-determine-the-cumulative-distribution-function.}{%
\subparagraph{b. Determine the cumulative distribution
function.}\label{b.-determine-the-cumulative-distribution-function.}}

\(\displaystyle F(X) = \int_{0}^x f(t) dt = \int_0^x \frac{3}{4}x(2-x)dt = \frac{3}{4} \int_0^x2t - t^2\)

\(\displaystyle \frac{3}{4}(t^2 - \frac{t^3}{3}) = \frac{3}{4}x^2(1-\frac{x}{3})\)

\(\therefore F(X) = \left\{ \begin{array}{ll} 0 & if \enspace x <0 \\ \frac{3}{4}x^2(1-\frac{x}{3}) & if \enspace 0 \leq x \leq 2 \\ 1 & if \enspace x > 2 \end{array} \right.\)

\hypertarget{c.-calculate-the-expectation-and-variance-of-x-.}{%
\subparagraph{c.~Calculate the expectation and variance of X
.}\label{c.-calculate-the-expectation-and-variance-of-x-.}}

\(\displaystyle E(X) = \int_{-\infty}^\infty x f(x)dx = \frac{3}{4} \int_0^2 (2x^2 - x^3)dx = \frac{3}{4} \cdot \left[ \frac{2x^3}{3} -\frac{x^4}{4}\right]_0^2 =\)

\(\displaystyle \frac{3}{4} \cdot \left( \frac{16}{3} - 4\right) = \frac{3}{4} \cdot \frac{4}{3} = 1\)

\(\displaystyle E(X^2) = \int_{-\infty}^\infty x f(x)dx = \frac{3}{4} \cdot \left[ \frac{2x^4}{4} -\frac{x^5}{5}\right]_0^2 = \frac{3}{4} \cdot \left( \frac{32}{4} - \frac{32}{5} \right) = \frac{6}{5}\)

\(\displaystyle Var(X) = \frac{6}{5} - 1^2 = \frac{1}{5}\)

\hypertarget{d.-use-tschebyschevs-inequality-to-determine-the-probability-that-x-does-not-deviate-more-than-0.5-from-its-expectation.}{%
\subparagraph{d.~Use Tschebyschev's inequality to determine the
probability that X does not deviate more than 0.5 from its
expectation.}\label{d.-use-tschebyschevs-inequality-to-determine-the-probability-that-x-does-not-deviate-more-than-0.5-from-its-expectation.}}

\(P(|X - \mu| < 0.5) \geq 1 - \frac{\frac{1}{5}}{0.5^2} = 1 - 0.8 = 0.2\)

\hypertarget{exercise-7.5}{%
\subsection{Exercise 7.5}\label{exercise-7.5}}

Consider the joint PDF for the type of customer service
\(X (0 = telephonic \enspace hotline, \enspace 1 = Email)\) and of
satisfaction score
\(Y (1 = unsatisfied, \enspace 2 = satisfied, \enspace 3 = very \enspace satisfied)\):

\begin{longtable}[]{@{}llll@{}}
\toprule
X/Y & 1 & 2 & 3 \\
\midrule
\endhead
0 & 0 & 1/2 & 1/4 \\
1 & 1/6 & 1/12 & 0 \\
\bottomrule
\end{longtable}

\hypertarget{a.-determine-and-interpret-the-marginal-distributions-of-both-x-and-y}{%
\subparagraph{a. Determine and interpret the marginal distributions of
both X and
Y}\label{a.-determine-and-interpret-the-marginal-distributions-of-both-x-and-y}}

\begin{longtable}[]{@{}ll@{}}
\toprule
X & \(P(X = x_i)\) \\
\midrule
\endhead
0 & 3/4 \\
1 & 1/4 \\
\bottomrule
\end{longtable}

\begin{longtable}[]{@{}ll@{}}
\toprule
Y & \(P(Y = y_i)\) \\
\midrule
\endhead
1 & 1/6 \\
2 & 7/12 \\
3 & 1/4 \\
\bottomrule
\end{longtable}

\hypertarget{b.-calculate-the-75-quantile-for-the-marginal-distribution-of-y}{%
\subparagraph{\texorpdfstring{b. Calculate the 75 \(\%\) quantile for
the marginal distribution of
Y}{b. Calculate the 75 \textbackslash\% quantile for the marginal distribution of Y}}\label{b.-calculate-the-75-quantile-for-the-marginal-distribution-of-y}}

The 75 \(\%\) have to follow:

\(F(x_p)\geq p\) \(F(x) < p \enspace for \enspace x < x_p\)

If we consider those properties, we will found that the third quartile
is \(X=2\) \(F(x = 2) = 1/6 + 7/12 = 9/12\) \(F(x=2) \geq 0.75\)

\hypertarget{c.-determine-and-interpret-the-conditional-distribution-of-satisfaction-level-for-x-1.}{%
\subparagraph{\texorpdfstring{c.~Determine and interpret the conditional
distribution of satisfaction level for
\(X = 1\).}{c.~Determine and interpret the conditional distribution of satisfaction level for X = 1.}}\label{c.-determine-and-interpret-the-conditional-distribution-of-satisfaction-level-for-x-1.}}

The conditional distribution is given as follows:
\(\displaystyle P(Y = y_i | X = 1) = p_{j|i} = \frac{p_{ij}}{p_{1+}} = \frac{p_{ij}}{1/4}\)

\(P(Y = 1 | X = 1) = \frac{1/6}{1/4} = 2/3\)

\(P(Y = 2 | X = 1) = \frac{1/12}{1/4} = 1/3\)

\(P(Y = 3 | X = 1) = \frac{0}{1/4} = 0\)

The customers using email like customer services are more uncomfortable.
The most of them have a level of satisfaction poor (they are
unsatisfied)

\hypertarget{d.-are-the-two-variables-independent}{%
\subparagraph{d.~Are the two variables
independent?}\label{d.-are-the-two-variables-independent}}

To be independent they have to follow:
\(P(X = x_i, Y = y_j) = P(X=x_i)P(Y = y_i)\)

\(P(X = 0, Y = 1) = 3/4 \times 1/6 \neq 0\)

\hypertarget{e.-calculate-and-interpret-the-covariance-of-x-and-y-.}{%
\subparagraph{\texorpdfstring{e. Calculate and interpret the covariance
of \(X\) and \(Y\)
.}{e. Calculate and interpret the covariance of X and Y .}}\label{e.-calculate-and-interpret-the-covariance-of-x-and-y-.}}

\(Cov(X,Y) = E(XY) - E(X)E(Y)\)

\(E(XY) = \sum_i \sum_j x_iy_ip_{ij}\)

\(\displaystyle E(X) = \sum\limits_{i=1}^kx_ip_i = x_1P(X=x_1)+x_2P(X=x_2)+ ... + x_kP(X=x_k)\)

\(E(X) = 0 \cdot 3/4 + 1 \cdot 3/12 = 1/4\)

\(E(Y) = 1 \cdot 1/6 + 2 \cdot 7/12 + 3 \cdot 1/4 = \frac{25}{12}\)

\(E(XY) = 0 \cdot 1 \cdot 0 + 0 \cdot 2 \cdot 1/2 + 0 \cdot 3 \cdot 1/4 + 1 \cdot 1 \cdot 1/6 + 1\cdot 2\cdot 1/12 + 1 \cdot 3\cdot 0 = \frac{2}{6}\)

\(Cov(X,Y) = \frac{2}{6} - \frac{25}{12} \cdot \frac{1}{4} =\) -0.1875

The value of covariance between X and Y is negative, so we could say
that high values of X have low values of Y.

\hypertarget{exercise-7.6}{%
\subsection{Exercise 7.6}\label{exercise-7.6}}

Consider a continuous random variable X with expectation 15 and variance
4. Determine the smallest interval \([15 − c, 15 + c]\) which contains
at least 90 \(\%\) of the values of X .

\(P(|X -15 | < c) = 0.9 \geq 1 - \frac{4}{c^2}\)

\(c = \sqrt{\dfrac{4}{0.1}}\)

\(c = \pm\) 6.3245553

\([15 -\) 6.3245553 \(, 15 +\) 6.3245553 \(]\)

\hypertarget{exercise-7.7}{%
\subsection{Exercise 7.7}\label{exercise-7.7}}

Let X and Y be two random variables for which only 6 possible events---
\(A1, A2, A3, A4, A5, A6\)---are defined:

\begin{longtable}[]{@{}lllllll@{}}
\toprule
i & 1 & 2 & 3 & 4 & 5 & 6 \\
\midrule
\endhead
\(P(A_i)\) & 0.3 & 0.1 & 0.1 & 0.2 & 0.2 & 0.1 \\
\(X_i\) & -1 & 2 & 2 & -1 & -1 & 2 \\
\(Y_i\) & 0 & 2 & 0 & 1 & 2 & 1 \\
\bottomrule
\end{longtable}

\hypertarget{a.-what-is-the-joint-pdf-of-x-and-y}{%
\subparagraph{\texorpdfstring{a. What is the joint PDF of \(X\) and
\(Y\)?}{a. What is the joint PDF of X and Y?}}\label{a.-what-is-the-joint-pdf-of-x-and-y}}

\begin{longtable}[]{@{}llll@{}}
\toprule
X /Y & 0 & 1 & 2 \\
\midrule
\endhead
-1 & 0.3 & 0.2 & 0.2 \\
2 & 0.1 & 0.1 & 0.1 \\
\bottomrule
\end{longtable}

\hypertarget{b.-calculate-the-marginal-distributions-of-x-and-y}{%
\subparagraph{\texorpdfstring{b. Calculate the marginal distributions of
\(X\) and
\(Y\)}{b. Calculate the marginal distributions of X and Y}}\label{b.-calculate-the-marginal-distributions-of-x-and-y}}

\(P(X_{-1}) = 0.7\)

\(P(X_2) = 0.3\)

\(P(Y_0) = 0.4\)

\(P(Y_1) = 0.3\)

\(P(Y_2) = 0.3\)

\hypertarget{c.-are-both-variables-independent}{%
\subparagraph{c.~Are both variables
independent?}\label{c.-are-both-variables-independent}}

To be independent they have to follow:
\(P(X = x_i, Y = y_j) = P(X=x_i)P(Y = y_i)\)

\(P(X=x_{-1}, Y = y_0) = 0.7 \times 0.4 \neq 0.3\)

\hypertarget{d.-determine-the-joint-pdf-for-u-x-y}{%
\subparagraph{\texorpdfstring{d.~Determine the joint PDF for
\(U = X + Y\)}{d.~Determine the joint PDF for U = X + Y}}\label{d.-determine-the-joint-pdf-for-u-x-y}}

-1 \textbar{} 0 \textbar{} 1 \textbar{} 2 \textbar{} 3 \textbar{} 4

P(U) \textbar{} 0.3 \textbar{} 0.2 \textbar{} 0.2 \textbar{} 0.1
\textbar{} 0.1 \textbar{} 0.1

\hypertarget{e.-calculate-eu-and-varu-and-compare-it-with-ex-ey-and-varx-vary-respectively.}{%
\subparagraph{\texorpdfstring{e. Calculate \(E(U)\) and \(Var(U)\) and
compare it with \(E(X) + E(Y)\) and \(Var(X) + Var(Y)\),
respectively.}{e. Calculate E(U) and Var(U) and compare it with E(X) + E(Y) and Var(X) + Var(Y), respectively.}}\label{e.-calculate-eu-and-varu-and-compare-it-with-ex-ey-and-varx-vary-respectively.}}

\(E(U)= -1 \cdot 0.3 + 0 \cdot 0.2 + 1 \cdot 0.2 + 2 \cdot 0.1 + 3 \cdot 0.1 + 4 \cdot 0.1 =\)
0.8

\(E(U^2) = (-1)^2 \cdot 0.3 + 0^2 \cdot 0.2 + 1^2 \cdot 0.2 + 2^2 \cdot 0.1 + 3^2 \cdot 0.1 + 4^2 \cdot 0.1 =\)
3.4

\(Var(U) = E(X^2)-[E(X)]^2 = 3.4 - 0.8^2 =\) 2.76

\(E(X) = - 1 \cdot 0.7 + 2 \cdot 0.3 =\) -0.1

\(E(X^2) = - 1^2 \cdot 0.7 + 2^2 \cdot 0.3 =\) 1.9

\(Var(X) = E(X^2)-[E(X)]^2 =\) 1.89

\(E(Y) = 0 \cdot 0.4 + 1 \cdot 0.3 + 2 \cdot 0.3\) 0.9

\(E(Y^2) = 0^2 \cdot 0.4 + 1^2 \cdot 0.3 + 2^2 \cdot 0.3\) 1.5

\(Var(Y) = E(Y^2)-[E(Y)]^2 =\) 0.69

\end{document}
